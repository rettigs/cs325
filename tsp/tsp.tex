\documentclass{article}
\usepackage{amssymb}
\usepackage[utf8]{inputenc}
\usepackage{geometry}
\usepackage{mathtools}
\usepackage{verbatim}

\geometry{letterpaper, portrait, margin=1in}

\title{CS325 - TSP Project}
\author{Group \#6 \\ William Jernigan, Alexander Merrill, Sean Rettig}
\date{\today}

\begin{document}
\maketitle
\part*{Our technique}
After trying a few algorithms, we realized that we had two separate types of algorithms: those which generated a tour (generator: e.g. nearest neighbor) and those which filtered through the tour and improved the cost of the tour (filter: e.g. genetic modification). Here we present our best combination of generator and filter, then the other generators and filters we tried.
\part*{Our best generator and our best filter}
\section*{growinject + inject3?}%growinject and inject3

\part*{Other generators}

\section*{Nearest Neighbor}%nn
As one of our earlier algorithms, nearest neighbor ran fairly quickly but produced suboptimal tours in comparison to most of our other generators. This generator always started its tour at the first city given in the input. Asymptotic runtime: $\Theta (n^2)$

\section*{Best of Nearest Neighbor}%nnbest
Since nearest neighbor always started with the first given city, we thought that varying the start city would produce better results. Nearest neighbor can produce a different result based on the starting city, so we ran nearest neighbor for all possible starting cities and selected the best result. Asymptotic runtime: $\Theta (n^3)$

\section*{Most common paths from Nearest Neighbor}%nncommon
This generator first gets all the nearest neighbor tours generated by using each city as a starting point. It then creates a graph where the weights of the edges are the number of times a path between the two cities occurred in all the nearest neighbor tours. Finally, for each possible starting city, it creates a tour by greedily taking the path most travelled from the current city to another city. The best tour from that final step is selected. Asymptotic runtime: $\Theta (n^3 * n^2 * n^3) = \Theta (n^7)$. Eeek. (it wasn't very good).

\section*{Greedy Cluster Merge}%gcm
We got this idea from group 14. Cities which are closest together are clustered. Then the distance from those clusters to other clusters are approximated and the nearest are clustered once again. This is continuted until the entire tour is constructed. Asymptotic runtime: $\Theta ()$ 
\section*{Permutations brute force}%brute

\section*{Injection}%inject
This filter takes a city, breaks it's existing edges and connects those two cities together, then injects the now free city it in an edge between two other points. Then the new path length is compared to the old path length and if it's shorter, the new path is kept. We iterate over every edge with every city until we cannot make any improvements in a full cycle.

\section*{Swap}%swap
Swap takes two cities in a path, swaps their order, then checks whether the resulting path is shorter and to be kept. Initially we did this by swapping neighbors, that is to say in the path a,b,c,d we swap a and b to get b,a,c,d, and iterating over the entire path. Once we got that working, we implemented this genetically (explained later) then iteratively where every point is checked against every other point until there is no path improvement in an entire cycle. Swap is fundimentally a subset of Injection and has been less effective in the sample sets.

\section*{Genetic}
Genetic algorithms take existing functions and apply them randomly to a dataset created by a generator to try to come up with better solutions. It ends up being functional and resulting in better paths but it's ultimately non-deterministic and therefore not optimal. We made genetic implementations of both our swap function and our injection function.

\section*{Grow Inject}%growinject
This takes the raw cities and 

\part*{Other filters}

\part*{Results}
INSERT GRAPH OF TOUR LENGTHS PRODUCED BY GENERATOR+FILTER COMBOS


\part*{Sources}


\end{document}
